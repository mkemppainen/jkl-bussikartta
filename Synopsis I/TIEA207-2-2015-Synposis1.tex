\documentclass[a4paper, twoside, finnish, english, 12pt]{article}
\usepackage[english,finnish]{babel}
\usepackage[T1]{fontenc}
\usepackage[utf8]{inputenc}
\usepackage[fixlanguage]{babelbib}
\usepackage{mathtools}
\usepackage{amsmath}
\usepackage{amsfonts}
\usepackage{amssymb}
\usepackage{amsthm}
\usepackage{icomma}
\usepackage{graphicx}
\usepackage{appendix}
\usepackage[center, it]{caption}
\usepackage{verbatim}
\usepackage{enumerate}
\usepackage{setspace}
\usepackage{fullpage}
\usepackage{booktabs}
\usepackage{comment}
\usepackage{dsfont}
%\usepackage{cmll} %Ei työkoneella
\usepackage{leftidx}
\usepackage{bbold}
\usepackage{todonotes} %Ei työkoneella
%\usepackage{SIunits}
\usepackage{units}
\usepackage{hyperref}
\usepackage{multirow}
\usepackage{fancyhdr}
\usepackage[noindentafter,nobottomtitles,toctitles]{titlesec}
\usepackage[nottoc]{tocbibind}
\usepackage{tocloft}
\usepackage{subcaption} %kuvakollaasin tekoon
\usepackage[section]{placeins}
%\usepackage{bibliography}

\title{}
\author{Joakim Linja}
%\numberwithin{equation}{section}

\hypersetup{%
	pdftitle = {TIEA207 Synopsis 1},                                 % tutkielman otsikko
	pdfauthor = {Linja Joakim Johannes , Ilmonen Kasimir Jyry Santeri, Homanen Mikko Aleksanteri, Kemppainen Mikko Taito Antero},                                % tekijä
	pdfsubject = {TIEA207},                                     % dokumentin tyyppi
	pdfproducer = {Linja Joakim Johannes , Ilmonen Kasimir Jyry Santeri, Homanen Mikko Aleksanteri, Kemppainen Mikko Taito Antero}     % Julkaisija
}

\begin{document}
\setlength{\parindent}{0pt} %Kommentoi englanninkielisessä dokumentissa
\setlength{\parskip}{4mm} %oli 4mm %Kommentoi englanninkielisessä dokumentissa
\onehalfspacing
\newcommand{\pound}{\operatornamewithlimits{\#}}
\newcommand{\lA}{\left\langle}
\newcommand{\rA}{\right\rangle}
\newcommand{\lI}{\left|}
\newcommand{\rI}{\right|}
\newcommand{\Unit}[1]{\ \unit{#1}}
\newcommand{\Var}[2]{$#1\ \unit{#2}$}
\newcommand{\Varf}[3]{$#1\ \frac{\unit{#2}}{\unit{#3}}$}
\newcommand{\Eqen}{\text{.}}
\newcommand{\todoi}[1]{\todo[inline]{#1}}
%\newcommand{\11}{\operatornamewithlimits{\mathbb{1}}}
%\let\imath=\i
\newcommand{\cluster}[2]{$\text{#1}_{#2}$}
%\newcommand{\picsize2}{0.175}
\newcommand{\vuosi}{2014}
\newcommand{\question}[1]{\subsection*{Question {#1} solution}}
\newcommand{\labwork}[1]{\newpage \section{#1}}
\newcommand{\E}[1]{\cdot 10^{#1}}
\newcommand{\Et}[1]{\ \text{x}10^{#1}}
\newcommand{\EUnit}[2]{\E{#1}\Unit{#2}}
\newcommand{\FUnit}[2]{\ \frac{\unit{#1}}{\unit{#2}}}
\newcommand{\EFUnit}[3]{\E{#1}\frac{\unit{#2}}{\unit{#3}}}
\newcommand{\upd}{\text{d}}

%\renewcommand\thesection{\Alph{section}}
%\renewcommand\thesubsection{\thesection.\arabic{subsection}}

\begin{flushleft}
Joakim Linja
\hfill 
\textsf{joakim.j.linja@student.jyu.fi}\hfill
\\ Kasimir Ilmonen\hfill
\textsf{kasimir.j.s.ilmonen@student.jyu.fi}\hfill
\\ Mikko Homanen\hfill
\textsf{mikko.a.homanen@student.jyu.fi}\hfill
\\ Mikko Kemppainen\hfill
\textsf{mikko.t.a.kemppainen@student.jyu.fi}\hfill

\hfill \texttt{15.10.2015}
\end{flushleft}
\setcounter{page}{1}
\selectlanguage{finnish}

\begin{center}
\huge
{TIEA207 Synopsis I}
\end{center}
%\question{}

\section{Projektista yleisesti}
%\section{Tiivistelmä}
Tarkoituksena on tehdä sovellus, joka web--käyttöliittymän avulla näyttää reaaliaikaisesti Jyväskylän kaupungin alueella liikkuvien linja-autojen arvioidun sijainnin kartalla. 

%\section{Kohderyhmä}
Kohderyhmänä projektille ovat kaikki jotka käyttävät linja-autoja Jyväskylän kaupungin alueella ja muut, joita kiinnostaa Jyväskylän kaupungin alueen linja-autoliikenne. 

%\section{Miellekkyys}
Projektin mielekkyys muodostuu kuriositeetista, joka kohdentuu linja-autojen kulkemisen visualisointiin. 
Tarkoituksena on konkretisoida linja-autojen liikennöintiä kaupungin alueella. 

\section{Projektin resurssikäyttö ja rajoitteet}
Projektin resurssikäyttönä on Jyväskylän kaupungin tarjoama avoin linkkidata ja kartan piirrossa käytetään Mapboxin kartta-API:tä. 
%\section{Datan saatavuus ja käyttöehdot}
Jyväskylän kaupungin tarjoaman linkkidatan käytössä on vaatimuksena maininta sovellukseen, että sovelluksen datan lähteenä on Jyväskylän kaupunki. 
Mapboxin käytön rajoitteena on, ettei Mapboxin tarjoamia tietoja käytetä mihinkään laittomaan toimintaan. 
Lisäksi kartassa tulee olla ''Improve this map''-linkki ja maininta siitä, että kartta on Mapboxilta lähtöisin. 

%\section{Datan formaatti}
Linkkidata haetaan Jyväskylän kaupungin sivustolta \url{http://data.jyvaskyla.fi/}, josta sen saa .zip--pakettina. 
Kyseinen zip-paketti sisältää kasan tekstitiedostoja, jotka vaativat jatkokäsittelyä. 
%\section{Käytettävät teknologiat}
Kartan piirto suoritetaan Leafletin päälle rakennetulla kirjastolla Mapbox. 
Palvelinohjelmisto tehdään alustavasti pythonilla. %Riippuen moniajolukon ilmenemisestä 
Asiakaspalvelu suoritetaan pythonin ja javascriptin yhteistyössä. 
Ohjelmointikielien tarkempia versionumeroita ei ole vielä tämän dokumentin kirjoitusvaiheessa päätetty. 
Projektin tuotoksen on tarkoitus pyöriä palvelimella siten, että sovellukseen pääsee netistä käsin kiinni. 

\section{Projektin tavoitteet}
Minimitavoitteena haluamme saada aikaan web--käyttöliittymän, jonka kartalla näkyy jonkin tietyn linjan linja-autojen sijainnit arvioiduilla reaalisijainneillaan. 
Laajennettuna tavoitteena on 
\begin{itemize}
\item[-] Kunkin yksittäisen linja-auton reitin erillisväritys kyseisen linja-auton kuvaketta klikkaamalla. 
\item[-] Erilaisia käyttäjän saatavilla olevia suodattimia linja-autojen ja niiden reittien visualisoinniksi. 
\item[-] Linja-autopysäkkien sijainnit, sekä linja-autopysäkkien kautta kulkevien linja-autojen erillisväritys. 
\item[-] Kartan piirto linja-autoineen käyttäjän valitsemalla ajanhetkellä. 
\end{itemize}
Projektin arkkitehtuurisuunnitelma on esitetty kuvassa \ref{fig:arkkitehtuurisuunnitelma}. 

%\section{Arkkitehtuurinen suunnitelma}

\begin{figure}[h]
\centering
\includegraphics[width=12cm]{projektisuunnitelma.png}
\caption{Projektin arkkitehtuurisuunnitelma}
\label{fig:arkkitehtuurisuunnitelma}
\end{figure}

%\bibliographystyle{jyflkandi_maunuksela_defined2}%babunsrt-fl}
%\selectbiblanguage{finnish}
%%\selectlanguage{english}
%\bibliography{lahteet}
\end{document}